\documentclass{jarticle}
\usepackage{gaiyo}
% 情報・通信工学科,先端工学基礎課程の学生は
% gaiyo.sty を書き換える必要があります.

\begin{document}
\nendo{令和3年度}
\nyugaku_nendo{平成30年度} 
\bango{1810519}
\name{林 真由}
\kyoin{宇都 雅輝}
\daimoku{評価者特性の時間変動を考慮した\\項目反応モデル}

%% 題目が1行に収まらない場合は,\parbox を使うとよいです
% \daimoku{
%   \parbox{340pt}{
%     博士の異常な愛情
%     または私は如何にして心配するのを止めて水爆を愛するようになったか
%   }
% }

\begin{abstract}
    近年,大学入試や資格試験,教育評価などの場において,パフォーマンス評価は重要な役割を果たしている.一方で,パフォーマンス評価では,評価者の厳しさや一貫性の違い,各得点の使用傾向の差などにより,採点に偏りが生じ,受検者の能力を正確に測ることが難しいという問題がある.この問題を解決するために,評価者特性の影響を考慮して受検者の能力を推定できる項目反応モデルが多数提案されている.
    これらのモデルは評価者の基準が採点過程で変化しないという仮定のもと成り立っているが,多数の受験者を長時間かけて採点するような場合には,この仮定は成り立たないことがある.このような評価者の採点基準が採点過程で変化する特性は評価者ドリフト(Rater Drift)と呼ばれ,これを考慮したモデルについても提案がなされている.既存モデルでは,評価者の厳しさの初期値と傾きを表すパラメータを導入することで,評価者特性の時間変化を捉えるモデルとなっているが,このモデルでは,評価者特性の変化を直線的にしかとらえることができない.この問題を解決するために,本研究では,時間区分ごとの評価者の厳しさを推定する新しい項目反応モデルを提案する.また,シミュレーション実験と実データ実験を通して提案モデルの有効性を示す.
\end{abstract}
\writeall
\end{document}
